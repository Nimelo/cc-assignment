\section{Research in collaborative platforms} \label{s:introduction:research-in-collaborative-platforms}
	There were four different approach considered. They can be split into two groups: plugin based implementation and own. First group contains \gls{drupal} and \gls{gitlab}, the other considers own implementation using \gls{CMS} like \gls{liferay} and totally own approach written in various languages and frameworks.
	
	\subsection{\gls{drupal}}
		It's one of the most popular \gls{CMS} currently used in the world. Contains a lot of useful modules that can be add and used in several ways. In terms of requirements we can consider two of them: \gls{drupal-module-group} and one of the following modules for file management:
		\gls{drupal-module-openlucious} or \gls{drupal-module-filedepot}. The main feature of this solution is widespread access to the modules, which are maintainable for a very long time. The second feature is very handy administration and themes control of the sites. Problem in this solution is fact that there is dependency on the modules that are already created, of course there is a possibility to write own one or change existing one but that approach requires a lot of effort and deep knowledge in this particular system. Main modules of \gls{drupal} are tested well, but extensions are not, which cause a lot of errors. 
	
	\subsection{\gls{gitlab}}
		\gls{git} is one of the most popular collaborative platforms for developers all around the world. In this approach described will be a web-based interface for such platform. Basically \gls{gitlab} is a repository manager with wiki and issue tracking features. It is an open source solution used widely by people. \gls{gitlab} was split into two products \gls{gitlabce} and \gls{gitlabee}. Both of them contains the same core features such as projects, groups, permissions file managements and administrations. The main features is very advanced and open revision control system. \gls{gitlab} can be also self-hosted or spreaded application which supports \gls{AWS} solutions. The others worth mentioning features are easy and handy way to upload files to repository through command line via console/terminal. The main issue is solution's complexity. Even having the access to the public code repository of \gls{gitlab} it would be hard to change behavior of core functionalities. 
	\subsection{Own implementation}
		The main issue of this solution starts with the design problem, even if the requirements are pretty well described. Time consumed for designing, implementing and testing is too high to consider this approach as correct for this assignment. using externally developed \gls{CMS} would decrease to overall time cost, but still it would require to fit into new environment.
	\subsection{Decision}
		Final thoughts about collaborative platform leads to \gls{gitlab}, because it already contain all the features that are necessary to fulfill requirements. It is also very popular platform used all over the world, which means that the concern about product is also very high, which increases the stability of platform itself. \gls{git} provides a protocol/mechanism that can be wrapped in a various way, which in terms of the further developments of the platform may be very helpful. The last argument for this solution is a time spent on the configuration. In this case consumed time is almost negligible.