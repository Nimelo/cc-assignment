%\chapter{Results} \label{chp:results}
\chapter{Test--Cases} \label{chp:test-cases}
	Some of the basic emergency situations can be solved or handled with several ways using \gls{gitlab}. The following section will introduce some small part of the real life examples of unexpected failures with step by step handling solutions.
	
	\gls{AWS} ensure redundancy among all provided services such as \gls{RDS}, \gls{EFS}, \gls{EC} and others. Depending on particular service some of have form of the pipeline that transfers files or creates simple database backups.
	
	\section{Redundancy and consistency} \label{s:test-cases:redundancy-and-consistency}
\begin{testcase}{Crash of master \gls{SQLDBMS}}{Medium}{None}
	{
		 	Not specified	
	}
	\term{Objective}{To handle unexpected situation.}
	\term{Inputs}{Crash of master PostgreSQL database.}
	\term{Steps}
	{
		\begin{enumerate}
			\item Replace connection with slave (replication) database server
			\item Check if slave database is operative
			\item Try to restart/fix master database
			\item If master database server is unfixable then create a new \gls{RDS} instance, apply latest master db backup and turn on replication from slave server.
		\end{enumerate}
	}
	\term{Output}{Slave database is master database and new instance will be slave server.}
\end{testcase}

\begin{testcase}{\gls{git} file system crash}{High}{None}
	{
		Not specified
	}
	\term{Objective}{To handle unexpected situation.}
	\term{Inputs}{\gls{git} file system crash.}
	\term{Steps}
	{
		\begin{enumerate}
			\item Remount file paths to temporary one in every running instance (projects settings will be still saved in \gls{SQLDBMS})
			\item Fix file system and merge changes by merging directories.
		\end{enumerate}
	}
	\term{Output}{Merged directories of \gls{git} file system.}
\end{testcase}
\clearpage
\begin{testcase}{\gls{redis} crash}{High}{None}
	{
		Not specified
	}
	\term{Objective}{To handle unexpected situation.}
	\term{Inputs}{\gls{redis} crash}
	\term{Steps}
	{
		\begin{enumerate}
			\item Reconfigure instances to use stand-alone caching solution.
			\item Try to fix main \gls{redis} component
			\item Replace stand-alone instances solution to external one again.
		\end{enumerate}
	}
	\term{Output}{Caching system is up and running via external solution.}
\end{testcase}

\begin{testcase}{Instance crash}{Medium}{None}
	{
		Not specified
	}
	\term{Objective}{To handle unexpected situation.}
	\term{Inputs}{Crash of the instance}
	\term{Steps}
	{
		\begin{enumerate}
			\item If the instance will get crash than auto-scaling group connected to the load balancer will create a new one automatically.
		\end{enumerate}
	}
	\term{Output}{Replication of the instance.}
\end{testcase}

\begin{testcase}{Overload of work on web-server}{Medium}{None}
	{
		Not specified
	}
	\term{Objective}{To handle unexpected situation.}
	\term{Inputs}{Overload of work on a instance}
	\term{Steps}
	{
		\begin{enumerate}
			\item If CPU Utilization will reach $60\%$ for more than $5$ minutes then a new instance will rise and start working up to 4 in maximum.
		\end{enumerate}
	}
	\term{Output}{Automatically adjust of the instances.}
\end{testcase}

\begin{testcase}{Underload of work on web-server}{Low}{None}
	{
		Not specified
	}
	\term{Objective}{To handle unexpected situation.}
	\term{Inputs}{Underload of work on web-server}
	\term{Steps}
	{
		\begin{enumerate}
			\item If CPU Utilization will be less than $40\%$ for longer than $5$ minutes then auto-scaling group connected to the load balancer will reduce amount of instances by 1 with a limit of 1 running instance.
		\end{enumerate}
	}
	\term{Output}{Automatically adjust of the instances.}
\end{testcase}
%	\section{Platform tests} \label{s:test-cases:platform-tests}
\chapter{Platform User--Guide} \label{chp:platform-user-guide}
	\gls{gitlab} is a very huge platform with a lot of features that are not necessary to use in terms of requirements described in Section \ref{s:introduction:platform-requirements}. All of them are presented on the official web-side referencing to \cite{bib:gitlabce-documentation-overview}. According to the requirements following sections will briefly describe how to create projects, groups and show how to perform basic administration activities. 	
	
	There is always a owner of a project and group, which has the highest rights. Owner can determine members of the group/project and specify the rights among users. There are four basic roles: Guest, Reporter, Developer and Master. Each of them can be configured and used in a different way. More details are specified in \gls{gitlab} web-sides.
	\section{Projects guide} \label{s:platform-user-guide:projects}
	\section{Groups Guide} \label{s:platform-user-guide:groups} 
	\section{Administrations guide} \label{s:platform-user-guide:administrations}
	\begin{figure}[!htbp]
		\centering
		\includegraphics[width=1\textwidth]{img/ug-administartion/groups}
		\caption{text}
		\label{key}
	\end{figure}

	\begin{figure}[!htbp]
		\centering
		\includegraphics[width=1\textwidth]{img/ug-administartion/projects}
		\caption{text}
		\label{key}
	\end{figure}

	\begin{figure}[!htbp]
		\centering
		\includegraphics[width=1\textwidth]{img/ug-administartion/monitoring}
		\caption{text}
		\label{key}
	\end{figure}

	\begin{figure}[!htbp]
		\centering
		\includegraphics[width=1\textwidth]{img/ug-administartion/overview}
		\caption{text}
		\label{key}
	\end{figure}

	\begin{figure}[!htbp]
		\centering
		\includegraphics[width=1\textwidth]{img/ug-administartion/users}
		\caption{text}
		\label{key}
	\end{figure}
