%\chapter{Results} \label{chp:results}
\chapter{Test--Cases} \label{chp:test-cases}
	Some of the basic emergency situations can be solved or handled with several ways using \gls{gitlab}. The following section will introduce some small part of the real life examples of unexpected failures with step by step handling solutions.
	\section{Redundancy and consistency} \label{s:test-cases:redundancy-and-consistency}
%	\section{Platform tests} \label{s:test-cases:platform-tests}
\chapter{Platform User--Guide} \label{chp:platform-user-guide}
	\gls{gitlab} is a very huge platform with a lot of features that are not necessary to use in terms of requirements described in Section \ref{s:introduction:platform-requirements}. All of them are presented on the official web-side referencing to \cite{bib:gitlabce-documentation-overview}. According to the requirements following sections will briefly describe how to create projects, groups and show how to perform basic administration activities. 	
	
	There is always a owner of a project and group, which has the highest rights. Owner can determine members of the group/project and specify the rights among users. There are four basic roles: Guest, Reporter, Developer and Master. Each of them can be configured and used in a different way. More details are specified in \gls{gitlab} web-sides.
	\section{Projects guide} \label{s:platform-user-guide:projects}
	\section{Groups Guide} \label{s:platform-user-guide:groups} 
	Groups similarly to projects have one owner. User can create a group from the left expander as shown in Figure \ref{fig:creating-group-view}. After clicking on the \emph{New Group} button user will be redirected to the view where he/she would be able to specify name, description and visibility of the group -- Figure \ref{fig:group-settings-view}. After creating the group user can add members with chosen permission similarly to the project -- Figure \ref{fig:adding-user-to-group} or add project as shown in Figure \ref{fig:adding-project-to-group}. Creating projects for the group looks exactly the same and is described in Section \ref{s:platform-user-guide:projects}.
	\section{Administrations guide} \label{s:platform-user-guide:administrations}
	\gls{gitlab} contains a lot of administration features that are described deeply in \cite{bib:gitlabce-documentation-overview}. One of the most interesting features is overview panel (Figure \ref{fig:administration-overview-view}) from which we can find information about latest projects, users, groups, statistics, features and components. Following three Figures \ref{fig:user-administration-view}, \ref{fig:groups-administration-view} and \ref{fig:projects-administration-view} presents administration panel which have functionality to add/delete/modify or temporary lock respectively users, groups and projects. The last worth mentioning view shows how hardware resources are used by application -- Figure \ref{fig:resource-usage-view}. This view shows the \emph{CPU}, memory and disks usage.
